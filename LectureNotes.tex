%%%%%%%%%%%%%%%%%%%%%%%%%%%%%%%%%%%%%%%%%
% Thin Sectioned Essay
% LaTeX Template
% Version 1.0 (3/8/13)
%
% This template has been downloaded from:
% http://www.LaTeXTemplates.com
%
% Original Author:
% Nicolas Diaz (nsdiaz@uc.cl) with extensive modifications by:
% Vel (vel@latextemplates.com)
%
% License:
% CC BY-NC-SA 3.0 (http://creativecommons.org/licenses/by-nc-sa/3.0/)
%
%%%%%%%%%%%%%%%%%%%%%%%%%%%%%%%%%%%%%%%%%

%----------------------------------------------------------------------------------------
%	PACKAGES AND OTHER DOCUMENT CONFIGURATIONS
%----------------------------------------------------------------------------------------

\documentclass[a4paper, 12pt]{article} % Font size (can be 10pt, 11pt or 12pt) and paper size (remove a4paper for US letter paper)
\usepackage[top=1in, bottom=1.25in, left=1in, right=1in]{geometry}
\usepackage[protrusion=true,expansion=true]{microtype} % Better typography
\usepackage{graphicx} % Required for including pictures
\usepackage{wrapfig} % Allows in-line images
\usepackage{caption}
\usepackage{amsmath}
\usepackage{mathtools}
\usepackage{float}
\usepackage{mathpazo} % Use the Palatino font
\usepackage[T1]{fontenc} % Required for accented characters
\linespread{1.05} % Change line spacing here, Palatino benefits from a slight increase by default

\makeatletter
\renewcommand\@biblabel[1]{\textbf{#1.}} % Change the square brackets for each bibliography item from '[1]' to '1.'
\renewcommand{\@listI}{\itemsep=0pt} % Reduce the space between items in the itemize and enumerate environments and the bibliography

\renewcommand{\maketitle}{ % Customize the title - do not edit title and author name here, see the TITLE block below
\begin{flushright} % Right align
{\LARGE\@title} % Increase the font size of the title

\vspace{50pt} % Some vertical space between the title and author name

{\large\@author} % Author name
\\\@date % Date

\vspace{40pt} % Some vertical space between the author block and abstract
\end{flushright}
}

%----------------------------------------------------------------------------------------
%	TITLE
%----------------------------------------------------------------------------------------

\title{\textbf{Large scale structure of the Universe}\\ % Title
 Lecture notes of the Andean Cosmology School} % Subtitle

\author{\textsc{Juan Nicol\'as Garavito Camargo \\ Lecturer: Yehuda Hoffman} % Author
%\\{\textit{Departamento de F\'isica\\}
%\textit{Universidad de los Andes, Bogot\'a, Colombia}}} % Institution

\date{1-26, June, 2015} % Date

%----------------------------------------------------------------------------------------

\begin{document}

\maketitle % Print the title section

\section{Cosmology review}

%----------------------------------------------------------------------------------------

The starting point is the Einstein equation 

\begin{equation}\label{eq:Einstein}
R_{\mu \nu} - \dfrac{1}{2}Rg_{\mu \nu} = 8 \pi G T_{\mu \nu} + \Lambda g_{\mu \nu}
\end{equation}

In which at the right hand is the geometry and in the left hand side
is the matter. With  the assumption that the Universe 
is homogeneous and isotropic at large scales the line element $dS$ is 
defined as:  

\begin{equation}
dS^2 = c^2 dt^2 - R(t)^2[\dfrac{dr^2}{1 - kr^2} + r^2(d\theta^2 - sin^2\theta d\phi^2)]
\end{equation}

Replacing the line element in the Einstein equation we can arrive to:

\begin{equation}\label{eq:R}
\dfrac{d^2 R}{d t^2} = - \dfrac{4 \pi G}{r } (\rho + \dfrac{3P}{c^2})
\end{equation}

Where  $R$ is the scale factor of the Universe.

In order to solve Eq.\ref{eq:R} we need to know what is the presure $P$
and the density $\rho$ in relativistic terms. The density can be expressed
as: 

\begin{equation}
\rho c^2 = \rho_0 c^2 + \epsilon
\end{equation}
Where $\rho_0 c^2$ is the rest frame density and $\epsilon$ is the 
internal energy per unit volume. 

To derive the Friedman equation we need to interate over the 
energy Eq.\ref{eq:R}. \textbf{Assuming that the expansion of the Universe
is adiabatic we have:}

\begin{equation}
dE = -pdV 
\end{equation}

\begin{equation}
dE = d(\dfrac{4\pi R^3 \epsilon}{3}) = - pdV = -p 4\pi R^2dR
\end{equation}

\begin{equation}
\dfrac{dE}{dt} = \dfrac{4\pi R^2 \epsilon \dot R}{3} + 
\dfrac{4\pi}{3}R \dfrac{d}{dt}(R^2 \cdot \epsilon) = - 4\pi \rho ***
\end{equation}

Now multiplying Eq.\ref{eq:R} by $\dot{R}$ we get:

\begin{equation}
\dot{R} \dfrac{d^2}{dt^2} R = \dfrac{1}{2}(\dfrac{dR}{dt})^2 = 
- \dfrac{4\pi}{3}\dfrac{G}{c^2}R\dot{R}(\epsilon + 3p)
\end{equation}

\begin{equation}
\dfrac{1}{2}\dfrac{d}{dt}(\dfrac{dR}{dt})^2 = \dfrac{4\pi}{3}\dfrac{G}{c^2}\dfrac{d}{dt} (R^2 \epsilon)
\end{equation}

\begin{equation}\label{eq:r2}
(\dfrac{dR}{dr})^2 = \dfrac{G}{R c^2}(\dfrac{4\pi}{3}R^2 \epsilon) + conts
\end{equation}

How is the enrgy density $\epsilon$ changing in time 
in order to solve Eq.\ref{eq:r2}

Newtonian analogue:
\begin{equation}
\dfrac{1}{2}\dot{R} - \dfrac{GM}{R} = conts
\end{equation}


The Friedman equation is:

\begin{equation}
\dfrac{\dot R ^2}{R^2} = \dfrac{8\pi G}{3}\rho - \dfrac{k}{R^2}
\end{equation}

\begin{equation}
\rho = {\epsilon}{c^2}
\end{equation}

\begin{equation}
\dfrac{\dot{R}}{R} = H 
\end{equation}


\begin{equation}
\dfrac{k}{H^2 R^2} = \dfrac{8 \pi G \rho}{3H^2} - 1 = \Omega -1
\end{equation}

Where $\Omega$ is the density parameter.

\begin{equation}
\rho_{crit}(t) = \dfrac{3 H^2}{8 \pi G}
\end{equation}

\begin{equation}
\Omega = \dfrac{\rho}{\rho_{crit}}
\end{equation}

\[
k >0, 0, <0 \rightarrow \Omega >1, 1 <1
\]

\textbf{$k$ remains constant in the Universe!?} Depending on $k$
we find different solutions for $R$.

In structure formation we are not concerning in the geometry of the
Universe. In our Universe $\Omega=1$ flat Universe. 

In a first approximation $R_0$ in the present time 
can be expressed as:

\begin{equation}
R(t) = R_0 + (\dfrac{dR}{dt})_0t + \dfrac{1}{2}\ddot{R}_0t^2
\end{equation}

\begin{equation}
\dfrac{dR}{dt} \rightarrow H = \dfrac{\dot{R}}{R} 
\end{equation}

and the second derivative. 
\begin{equation}
\dfrac{d^2R}{dt^2} \rightarrow q = - \dfrac{1}{H^2}\dfrac{\ddot R}{R}
\end{equation}

How the energy density changes with time to start looking to solutions
for the Friedman equation. Finding a equation state could be extremely 
difficult in some cases e.g Supernovae explotions. 

The simplest equation state that Cosmologist use is:

\begin{equation}
P = W\rho c^2
\end{equation}

The equation of state for Cold Dark Matter (CDM) (non relativistc
matter.)

\begin{equation}
W = 0 (no pressure, cold gas)
\end{equation}

\begin{equation}
W = 1/3 (Relativistc particles) e.g CMB or in the early Universe
\end{equation}

\begin{equation}
W = -1 (radiative dominated Universe) \Lambda
\end{equation}

Using this equation state in the Friedman equation 
we get. 

\begin{equation}
q_0 = \dfrac{\Omega_0}{2}(1 + \dfrac{3P}{\rho c^2})
\end{equation} 

\begin{equation}
q = \Omega_0 (1/2, 1, -1/2) \rightarrow W= 0, 1/3, 1
\end{equation}

\begin{equation}
dE = -pdV
\end{equation}

\begin{equation}
d(\epsilon R^3) = - p d(\epsilon R^3)
\end{equation}

\begin{equation}
\dfrac{d(\rho R^3)}{d R^3} = -w \rho
\end{equation}

\begin{equation}
\rho \alpha R^{-3(1+W)}
\end{equation}

From the Planck satellite we know that

\begin{equation}
\Omega_0 = \Omega_m + \Omega_R + \Omega_{\Lambda} = 1 
\end{equation}

When you have different components that dominate in the Universe 
you have to take into aacount the combiantion and the energy density 
goes as acombintation f the contributions. If we take into acccount 
the combination of the componentes we find that radiation and 
matter have the same amount of contrubition at redshift:

\begin{equation}
z_{eq} = 3371 \pm 23
\end{equation}

and the redshift at which the matter and the cosmoligical constant ar 
equal is 

\begin{equation}
z_{eq DM - \Lambda} \approx 1.4
\end{equation}

\subsection{Friedman lutions}

\subsubsection{Einsten de Sitter Universe}

Is the matter dominates Universe

\begin{equation}
\Omega_m = 1
\end{equation}


In the  Friedman equation we get:

\begin{equation}
\dfrac{H^2}{H_0 ^2} = (\dfrac{R_0}{R})^3 = (1+z)^3
\end{equation}

\begin{equation}
(\dfrac{\dot{R}}{R})^2 = H_0^2 (\dfrac{R_0}{R})^3
\end{equation}

\begin{equation}
\dot{R}^2 = H_0^2R_0^3 / R 
\end{equation}


\begin{equation}
\dfrac{2}{3} R^{3/2} = H_0 R_0^3 t 
\end{equation}

\begin{equation}
t_0 = \dfrac{2}{3}H_0^{-1}
\end{equation}

\begin{equation}
\dfrac{R}{R_0} = (\dfrac{t}{t_0} )^{2/3}
\end{equation}

\subsubsection{Closed matter dominated universe $\Omega_m$ > 1 }

\begin{equation}
\dot{R} = \dfrac{4\pi G \rho_0 R^3}{3R} - c^2
\end{equation} 

\begin{equation}
\dot{R} = 0  \rightarrow R = \dfrac{8\pi G \rho_0 R_0^3}{3c^2} = R_{max}
\end{equation}

\begin{equation}
R ... cos \theta
\end{equation}

poner plot de R vs T 

\begin{equation}
t_{max} = \dfrac{R_{max}}{4c}\pi
\end{equation}

\begin{equation}
R = 0 \rightarrow \theta = 2\pi
\end{equation}

\begin{equation}
t_{crunch} = 2 t_{max}
\end{equation}

\subsubsection{$\Omega < 1$}


\begin{equation}
\dot{R}^2 = \dfrac{8\pi G \rho_0 R_0^3}{3R}+ c^2
\end{equation}

\begin{equation}
R_{*} = \dfrac{4 \pi G \rho_0 R_0^3}{3c^2}
\end{equation}

\begin{equation}
R = R_{*} = (cosh \theta - 1)
\end{equation}

\begin{equation}
t = \dfrac{R_{*}}{c}(sinh \theta - \theta)
\end{equation}

\end{document}

